% Created 2010-03-26 Fri 23:30
\documentclass[11pt,a4paper]{article}
\usepackage[utf8]{inputenc}
\usepackage[T1]{fontenc}
\usepackage{hyperref}
\usepackage[english,nynorsk]{babel} % or whatever language
\usepackage{apacite} % after babel
\usepackage{natbib}
\usepackage{graphics}

\usepackage{amsmath} % for \operatorname

\usepackage{tree-dvips}
\usepackage{linguex} 

\usepackage{pslatex}
% \usepackage{pdfsync} % bug with glosses ( \exg. in linguex )
\pdfoutput=0

\usepackage{qtree}
\usepackage{avm}
\avmfont{\sc}
\avmoptions{sorted,active}
\avmvalfont{\rm}
\avmsortfont{\scriptsize\it}
\newcommand{\xbar}{$\rm\overline{X}$}

\title{Syntaktisk informert frasesamanstilling }
\author{Kevin Brubeck Unhammer}
\date{26/03, 2010}

\begin{document}

\maketitle

\section{Kapittel x: Den ideelle frasesamanstillinga}
\label{sec-1}

\subsection{Introduksjon}
\label{sec-1.1}

I denne delen prøver eg å finne fram til kva som er den best moglege
frasesamanstillinga. Eg argumenterer for at «best» her må tolkast i
forhold til eit formål, og tek utgangspunkt i visse krav for
ordsamanstilling gitt i \citet{thunes2003eal}. Eg kjem fram til at når
formålet er utvikling av fasesamanstilte trebankar må ein revidere
kravet om likskap i argumentstruktur, og gir eit forslag til krav for
frasesamanstilling i trebankar.

\subsection{Kva er formålet med ei frasesamanstilling?}
\label{sec-1.2}

I frasebasert statistisk maskinomsetjing (PBSMT) skal ei
fraselenkje\footnote{Eg nyttar her termane \emph{lenkjing} og \emph{samanstilling} om
 kvarandre, i same tyding som det engelske \emph{alignment}; dette er
 ekvivalensforhold som me kan finne mellom lingvistiske
 \emph{representasjonar} (f-struktur, c-struktur) eller \emph{uttrykk} (ord,
 setningar). Lenkjing mellom dei siste altså er meir ateoretisk / datanært. } forbetre maskinomsetjing på eitt eller anna mål,
t.d. \textsc{Bleu}-skåren. \textsc{Bleu}-skåren samanliknar ferdig
omsett tekst (ein gullstandard) med det automatisk omsette, ved å
sjekke kor mykje N-gram-overlapp det er mellom tekstene. Ei
fraselenkje mellom N-grammet \emph{es gibt} og \emph{there is} (dvs. eit auka
sannsyn for å nytte slike par i omsetjinga) kan gi ein høgare endeleg
skåre i \textsc{Bleu}. Som vist i \citet{koehn2003spb} fekk dei ein
lågare \textsc{Bleu}-skåre når dei fjerna lenkjer mellom nodar som, i
følgje ein robust statistisk PCFG-parser, ikkje var syntaktiske frasar
(konstituentar). Dvs. at i figur \ref{fig:ikkjenode} vil lenkja vist
ved den prikkete lenkja bli fjerna frå mengda over moglege lenkjingar
om ein berre held seg til syntaktiske konstituentar, og
$p(es~gibt,~there~is)$ vil ikkje bli tilsvarande auka i den
statistiske omsetjingsmodellen. Sidan PBSMT, som skildra i
\citet{koehn2003spb}, er agnostisk til syntaktiske høve i
omsetjingssteget\footnote{Både omsetjingsmodellen og
språkmodellane er reint N-grambaserte her, og har difor ikkje nytte av
syntaktisk informasjon (i motsetning til syntaktisk informert
generering slik \citet{riezler2006gmt} implementerer). } er det for dei ingen grunn til å berre halde
seg til samanstilling mellom syntaktiske konstituentar; dei har i
utgangspunktet meir nytte av kollokasjonsinformasjon.

\begin{figure}[htp]
 \vfill{} % how todo?
\Tree [ [.\node{aDE}Es \node{nDE}{} ]
        \qroof{\node{bDE}{gibt} Frost an meiner Tür }.\node{pDE}{XP} ]
\Tree [ [.\node{aEN}There \node{nEN}{} ]
        \qroof{\node{bEN}{is} frost at my door      }.\node{pEN}{YP} ]
 \nodecurve[t]{pDE}[t]{pEN}{30pt}
 \barnodeconnect[-12pt]{nDE}{bDE}
 \barnodeconnect[-5pt]{nEN}{bEN}
{\makedash{4pt}
 \nodecurve[b]{bDE}[b]{nEN}{15pt} 
}
 \caption{N-gram-samanstilling versus syntaktiske frasar}
 \label{fig:ikkjenode}
\end{figure}

Men sett no at me ikkje har som formål å nytte frasesamanstillinga til
reint N-grambasert omsetjing. Kva for \emph{lingvistiske} krav kan me stille
til å kalle to frasar samanstilte? I einkvar større parallelltekst vil
parallellstilte setningar ha visse syntaktiske og semantiske\footnote{Sidan eg føreset setningssamanstilte data, kjem eg ikkje inn på
 diskurs-/pragmatiske verknader, med mindre det kan vere mogleg
 å handsame desse innanfor setningen. }
omsetjingsskifte, t.d. leksikalisering av syntaktiske konstruksjonar
eller omvendt, endring av ordklasse, presisering/depresisering,
endringar i leksikale trekk (t.d. telleleg/utelleleg),
osb. \citep[s.~56--62]{munday2001its}, slik at den einaste
fullstendige, «perfekte» samanstillinga vil vere
identitetsfunksjonen. Me må godta ein del mangel på samsvar; kor mykje
me godtek blir då avgjort av formålet med samanstillinga.

Eg føreset her at eitt av formåla med samanstillinga er å kunne
oppdage korleis ulike språk realiserer semantiske roller syntaktisk;
då spesielt i forhold til hypotesane gitt i \citet[s.~7]{xpar2009pd},
t.d. at «case marking might be useful to further determine a given
argument's semantic role». (Skal me finne det siste, må me altså kunne
samanstille frasar med ulik kasusmarkering, men ha krav om lik
tildeling av semantiske roller.)

Eit anna mogleg formål er å nytte desse frasesamanstillingane til
maskinomsetjing. \citet{riezler2006gmt} nyttar ein stokastisk
frasesamanstilling til å oppdage transfer-reglar for bruk i LFG-basert
generering i maskinomsetjing. Dette er reglar som omsett fragment av
ein f-struktur på kjeldespråket til f-strukturfragment på
målspråket. (Eit krav på utforminga av moglege transfer-reglar hindrar
at ein får reglar som lenkjar ikkje-konstituentar, eg kjem tilbake til
dette nedanfor.)  Samanstillinga utvikla her burde au kunne nyttast
til å finne slike transfer-reglar.

Nedanfor utviklar eg eit forslag til krav for ei frasesamanstilling,
med desse formåla i tankane. Om alle krava er moglege å implementere,
er eit separat problem.

\subsection{Krav / skrankar for frasesamanstilling i ein LFG-trebank}
\label{sec-1.3}


Samanstilte frasar bør ha nok semantisk likskap til å kunne opptre som
omsetjingar i liknande omgivnader
\citep{dyvik2009pc}. \citet{thunes2003eal} gir nokre passande prinsipp
for å fastslå det som kan kallast \emph{omsetjingsmessig korrespondanse}, for
ordsamanstilling. Dette er prinsipp som skal gjelde for eit litt forskjellig
formål\footnote{\cite[s.~2]{thunes2003eal}: «Våre prinsipper er satt
opp for å tjene et bestemt formål, nemlig å samle inn data som metoden
i Semantic Mirrors skal anvendes på», ein metode for å automatisk
finne WordNet-liknande relasjonar frå parallelltekst. I denne metoden
vil det vere naturleg med høge krav til presisjon, men kanskje lågare
krav til dekning: speilmetoden skal finne leksikale semantiske forhold
som held på \emph{typenivå}, medan for trebanken er det viktigare korleis
me kan annotere eit \emph{token} av t.d. eit verb i ein viss VP i ei gitt
korpussetning. }, men som au «ligger nær opp til det vi intuitivt
mener er riktig» \citep[s.~2]{thunes2003eal}. Prinsippa blir nytta til
å lage ein gullstandard for ordsamanstilling (hovudsakleg for dei opne
klassene), og er definert ved å vise til kva for rolle eit argumentord
speler, eller kva for rolletildeling eit predikat eller modifiserande
ord gir. Så for å t.d. samanstille to verb må dei ha like mange
semantiske argument (men argumenta treng ikkje alle realiserast
syntaktisk) og dei må \emph{tildele same roller}; medan argumenta må \emph{spele same rolle}, og både argument og adjunkt må vere \emph{koreferente}. Lenkja
ord må vere del av frasar som speler same rolle i «det som er felles i
interpretasjonene av [dei to setningane]» \citep[s.~3]{thunes2003eal}.

Viss me tek utgangspunkt i det siste, vil det vere naturleg å i
tillegg lenkje desse frasane som speler same rolle i «det som er
felles i interpretasjonene».

Krava for ordsamanstillinga må au vere fylt for at desse frasane kan
samanstillast. Ein ordsamanstilling er altså naudsynt for ein
frasesamanstilling, og omvendt. Dette er berre motsetningsfylt om me
føreset at det eine er derivert av det andre; men dette har me ingen a
priori grunn til å gjere. Krava eg her utviklar bør i staden sjåast på
som \emph{skrankar} på moglege samanstillingar, på same måte som dei
modellteoretiske tolkingane av LFG og HPSG.

\citet{pullum2001dbm} gir ein god gjennomgang av forskjellen
mellom derivasjonelle (enumerative) grammatikkar og skrankebaserte
modellteoretiske grammatikkar, kor førstnemnde definerer \emph{mengder av uttrykk} ved avleiing frå startsymbol, medan sistnemnde gir skildringar
av \emph{enkeltuttrykk}. Ein modellteoretisk grammatikk kan i tillegg skildre
strukturen (eller dei moglege strukturane) til \emph{fragment} av setningar,
og denne strukturen er lik det bidraget som fragmentet tilfører
skildringa av heile setninga. Det tilsvarande er ikkje mogleg å gjere
derivasjonelt. \citet[s.~32--33]{pullum2001dbm} gir t.d. eit fragment
som kjem midt i eit høgreforgreina tre; ein derivasjonell skildring
ville måtte skildre treet over eller under, men utan informasjon om
kva som kjem til høgre eller venstre kan me ikkje (på ein
ikkje-vilkårleg måte) skildre subtreet utanfor fragmentet heilt fram
til terminal- eller startsymbol. 

Sidan ei frasesamanstilling er ei skildring av forhold mellom
setningsfragment vil det vere naturleg å skildre dei ønskelege
forholda som skrankar på moglege samanstillingar. Dette let oss au
setje skrankar på både frase- og ordsamanstilling sameleis, utan å
måtte ha krav om at den eine samanstillinga er fullstendig avleiia av
den andre; noko me ikkje har eit \emph{a priori} grunnlag for å seie. 

Sidan metoden er mynta på bruk i ein LFG-parsa trebank, og delvis vil
nytte denne parsen som datagrunnlag, er det naturleg å nytte same
konsept som blir nytta i LFG\footnote{I tillegg finst andre positive biverknader av ein LFG-basert
 frasesamanstilling for bruk i denne samanhengen, som at ein kan
 oppdage kor parallelle dei parallelle grammatikkane i
 ParGram-prosjektet \citep{butt2002pgp} faktisk er, på ulike nivå
 (leksikon og argumentstruktur, c-struktur, f-struktur). } (f-struktur, c-struktur,
endosentrisitetsprinsipp, \xbar{}-tre, osb.)  au i desse krava til den
«beste» frasesamanstillinga; i den grad LFG gir ein generaliserbar
skildring av syntaks, bør desse krava vere generaliserbare til andre
teoriar.

Eg byggjar vidare på krava frå \citet{thunes2003eal} nedanfor, men
kjem som nemnd med visse endringsforslag.

\subsection{Kva kan samanstillast?}
\label{sec-1.4}


Viss to uttrykk er samanstilt på setningsnivå (slik at me dimed kan gå
ut frå at dei er omsetjingar av kvarandre), og båe har ein
LFG-analyse, så har me iallfall tre ulike nivå kor me kan finne
ekvivalensforhold under setningsnivå:
\begin{enumerate}
\item mellom ord i setningane,
\item mellom f-strukturar,
\item mellom c-strukturnodar.
\end{enumerate}
Alle ord i setninga er \emph{kandidatar} for samanstilling med ord i
omsetjinga, men \emph{a priori} kan me ikkje utelukke at eit ord ikkje har ei
lenkjing, og me kan heller ikkje utelate mange-til-mange-lenkjing. Det
same gjeld nodane i c-strukturen.


Når det gjeld f-strukturane er det ganske mange element me teoretisk
sett kunne ha samanstilt, t.d. enkelttrekk som bestemtheit eller dei
uordna mengdene med adjunkt, men det som er mest \emph{nyttig} er nok å
berre gjere samanstillingar der det er ei nær kopling til orda i
setninga. Sidan alle PRED-element i ein f-struktur unikt står for
predikerande ord, kan me -- gitt to samanstilte setningar -- la
\emph{kandidatane for samanstilling på f-strukturnivå} inkludere\footnote{I del \ref{SEC:fnord} kjem eg tilbake til spørsmålet om me vil
        inkludere visse f-strukturar utan PRED-element i kandidatane
        for samanstilling. }
alle desse PRED-elementa i f-strukturane til setningane. PRED-element
representerer semantiske bidrag som oftare er naudsyne på båe språk i
omsetjingar, medan andre f-strukturtrekk gjerne er valfrie på det eine
av språka; det er ikkje alle språk som har t.d. obligatorisk
kasusmarkering, og ein vil kanskje nytte trebanken til å oppdage
nettopp slik variasjon.  PRED-elementa er i tillegg gjerne enklare å
knyte direkte opp mot konkrete tekststrengen, medan t.d. aspekt
kanskje er umogleg å skilje frå tempus i affikset.

Eg føreslår følgjande føringar:

\ex. \label{f-links} Ei samanstilling av to PRED-element i f-strukturane tilseier at:
\a. \label{f-links-substr} f-strukturane til desse er lenkja,
\b. \label{f-links-words} orda i setningane som projiserer
   PRED-elementa tek del i ei samanstilling med kvarandre (kor andre
   ord kan vere involvert), og at
\c. \label{f-links-domain} iallfall dei øvste nodane i det funksjonelle
   domenet\footnote{Det funksjonelle domenet til ein f-struktur er gitt ved
 $\phi^{-1}$, inversen av c-til-f-strukturavbildinga, og tilsvarer dei
 nodane i c-strukturen som projiserer denne f-strukturen, t.d. ein
 VP-node med dominerande IP og CP
 \citep[s.~126]{bresnan2001lfs}. Sidan dette er inversen av ein
 funksjon, kan me ha diskontinuerlege konstituentar i same
 funksjonelle domene (fleire funksjonsargument som gir same verdi). } til f-strukturen er samanstilt.

(Underordna nodar i det funksjonelle domenet kan berre lenkjast om
visse krav, gitt nedanfor, er oppfylt. Me kan altså gjerne ha
c-strukturnodar som ikkje er lenkja til andre nodar.)

Påstandane over må forsvarast. Punkt \ref{f-links-substr} og
\ref{f-links-domain} over seier at viss PRED-elementa projisert av
t.d. to verb i verbfrasar er lenkja, vil \emph{heile} VP-ane vere lenkja
(både VP-nodane som dominerer dei lenkja funksjonelle domena og
f-strukturane frå ytre PRED til verba), det er dette som gjer det til
ei fraselenkje; medan i følgje punkt \ref{f-links-words} vil denne
fraselenkja leie til at sjølve verba au er lenkja, ein sterkare
påstand sidan dette tilseier at \emph{PRED-samanstilling impliserer ordsamanstilling}. I visse tilfelle er dette heilt uproblematisk,
t.d. viss \emph{I slept down by the river} skal lenkjast med \emph{Eg sov nede med elva} vil me uansett lenkje \emph{slept} og \emph{sov}; dette kan gjelde
transitive verb au:

\ex. \a. The locusts have no king, just noise and hard language\\
     $\leftrightarrow$
     \b. Grashoppene har ingen konge, berre støy og krasse ord

\emph{have/har} tek del i VP-samanstillinga \emph{have no king.../har ingen konge...}.

Som nemnd over; ordsamanstillinga treng ikkje vere ein-til-ein, det
punkt \ref{f-links-words} seier er at desse orda iallfall er ein del
av ein samanstilling med kvarandre (i \Last altså
VP-samanstillinga). Kanskje er dette ei mange-til-mange-lenkjing som
ikkje \emph{kan} reduserast til ein-til-ein-lenkjingar; eller kanskje er
det som i \Last mogleg å skilje ut delsamanstillingar, som
\emph{have/har}. Eg kjem tilbake til dette i del \ref{SEC:lik-argstr} om
argumentstruktur og adjunkt. 


Alle nodar i c-strukturen (alle syntaktiske \emph{frasar/konstituentar} i
setninga) som kan koplast til PRED-haldande f-strukturar, vil altså
vere kandidatar for samanstilling på c-strukturnivå (dette inkluderer
diskontinuerlege konstituentar), men ikkje alle vil bli samanstilt.
\subsubsection{\textbf{TOGROK} finst det tilfelle der ordlenkjer ikkje impliserer PRED-lenkjer?}
\label{sec-1.4.1}

   hypotese: det er alltid slik at \\
   ordlenkjing av predikerande ord => PRED-lenkje
\subsection{Funksjonsord}
\label{sec-1.5}

\label{SEC:fnord}
I tillegg kan me ha ord i setninga som ikkje tilsvarer PRED-element i
f-strukturen, typisk funksjonsord (t.d. \emph{som}, \emph{at}). Ved
endosentrisitetsprinsippa til \citet{bresnan2001lfs} er komplementet
til funksjonelle kategoriar (C, I, P) ein funksjonell ko-kjerne. 

\ex. \label{fnordkrav} Skal nodar for ord som ikkje projiserer
     PRED-element\footnote{Skal ein lenkje ordet \emph{som} (utan PRED) med ordet \emph{which} (med
 PRED)? Viss båe står under C i treet, kan det kanskje vere
 informativt med ein type «defekt» lenkje, sjølv om berre det eine
 ordet blir rekna for å vere eit innhaldsord. Frasane til deira
 funksjonelle domene vil uansett vere samanstilt via toppnodane
 (t.d. CP). } samanstillast, må følgjande krav vere oppfylt:
\a. det funksjonelle domenet (gitt ved komplementet) må vere
   samanstilt, og
\b. dei er båe c-strukturhovud.


Om \Last[a og -b] er oppfylt, kan me få samanstillinga vist i figur
\ref{fig:fnord}, og i dette tilfellet er \Last[b] oppfylt og \Last[a]
vil vere oppfylt om me kan samanstille \emph{cvimda} med \emph{det regnet}.

\begin{figure}[htp]
 \vfill{} % how todo?
\Tree
[.IPfoc
  [.PROPP [.PROP abramsma ] ] 
  [.I' \qroof{iCoda}.I 
           [.S [.CPsub
                [.\node{Csub}{Csub} rom ]
                \qroof{cvimda}.IP ]]]]
\Tree
[.IP
  [.PROPP [.PROP Abrams ] ]
   [.I' [.Vfin visste ]
            [.S [.VPmain [.CPnom
                         [.\node{Cnom}{Cnom} at ] 
                          \qroof{det regnet}.Ssub ]]]]]
                         
{\makedash{4pt} \nodecurve[b]{Csub}[b]{Cnom}{30pt} }
\caption{Mogleg samanstilling av funksjonsord mellom georgisk og norsk (bokmål)}
 \label{fig:fnord}
\end{figure}
\subsection{Lenkjing av underordna c-strukturnodar}
\label{sec-1.6}

\label{SEC:subnode}

Toppnodane i eit lenkja funksjonelt domene i c-struktur (XP på språk
1, ZP på språk 2) vil ha ein informasjonsmessig korrespondanse, og kan
samanstillast. Men det er mogleg å samanstille to toppnodar i
funksjonelle domene i c-strukturen utan at nodane under (X', Z') er
samanstilt. Ein grunn til å ikkje samanstille desse underordna nodane,
vil vere viss spesifikator til X ikkje speler same rolle i tolkinga
som spesifikator til Z, dvs. viss YP og WP i figur \ref{fig:subnode}
ikkje er lenkja.


Me kan utelukke lenkjing av ikkje-konstituentar som \emph{there is} ved å
krevje at ei fullstendig samanstilling mellom to frasar må vere slik
at heile substrukturen au er samanstilt. \emph{There is} og \emph{Es gibt} i
figur \ref{fig:ikkjenode} kan då ikkje samanstillast åleine, men berre
som del av ei ytre frasesamanstilling.
Så når \emph{kan} me samanstille nodane som står under øvste node i
f-domenet?

\begin{figure}[htp]
 \vfill{} % how todo?
\Tree  [.\node{XP}{XP} \node{YP}{YP} 
                                  \node{X'}{X'}  ]
\Tree  [.\node{ZP}{ZP} \node{WP}{WP} 
                                  \node{Z'}{Z'}  ]
\barnodeconnect[5pt]{XP}{ZP}
{\makedash{4pt}
 \nodecurve[b]{YP}[b]{WP}{15pt} 
}
 \caption{Lenkjing av underordna c-strukturnodar}
 \label{fig:subnode}
\end{figure}

I figur \ref{fig:subnode} der XP og ZP er lenkja, vil YP og WP -- i
kraft av å vere toppnodar i sine domene -- måtte ha ei lenkje i
f-strukturen for at c-strukturnodane kan lenkjast (det kunne jo
t.d. hende at f-strukturen projisert av YP samsvarte med den projisert
av Z', eller ein struktur under Z').

Om me skal lenkje Z' og X' i figuren over må dei respektive
spesifikatornodane vere lenkja. Me får då følgjande krav:

\ex. \label{subnodekrav} Krav for lenkjing av underordna
c-strukturnodar:
\a. c-strukturnodar som ligg under øvste node i to funksjonelle
    domena kan berre samanstillast med nodar som ligg innanfor desse
    domena,
\b. c-strukturnodar kan berre samanstillast om deira funksjonelle
    domene er lenkja på f-strukturnivå,
\c. om ein c-strukturnode X' som ikkje er toppnode i det funksjonelle
    domenet har ein søsternode YP, må YP vere samanstilt med ein
    søsternode til Z' for å samanstille X' og Z'


\Last[a] seier at om XP og ZP er samanstilt, der XP er t.d. OBJ til
IP, kan ikkje Z' samanstillast med SUBJ til IP osb., men berre til
nodar innanfor OBJ-domenet. \Last[c] påført figur \ref{fig:subnode}
seier altså at spesifikatornodane må vere lenkja for at X' og Z' skal
lenkjast (manglande søsternode på den eine sida vil au hindre
samanstilling).

I figur \ref{fig:fnord} er alle nodane under S vist i dei to trea i
same funksjonelle domene (kvar node under S er annotert med $\uparrow
= \downarrow$), så om dei funksjonelle domena er samanstilt (som krev
at \emph{rom cvimda} og \emph{at det regner} er samanstilt), vil \Last[a og -b]
vere oppfylt kva gjeld CP-komplementa -- lenkjinga går ikkje ut over
dei funksjonelle domena. Sidan Csub og Cnom er funksjonelle kategoriar
er dei au samanstilt via samanstillinga av S-nodane og føringane i
\ref{fnordkrav}, og \Last[c] er då oppfylt. \Last står altså ikkje i
vegen for å samanstille IP-en over \emph{cvimda} og Ssub.

I figur \ref{fig:ikkjesub} derimot \citep{mrs-suite}, kan me ikkje
samanstille I'-nodane. PRONP-noden, spesifikator på den norske sida,
er ikkje lenkja med nokon spesifikator på den georgiske sida. Den
informasjonen (her reint syntaktisk) som ordet \emph{det} tilfører IP, ligg
under I' på georgisk. Om me skulle lenkja I', måtte me altså hatt ein
georgisk spesifikator som var lenkja til den norske PRONP.

\begin{figure}[htp]
 \vfill{} % how todo?
\Tree
[.\node{IPk}{IP}
  [.\node{Ibark}{I'} [.V    \node{gaiGo}{gaiGo} ]
  ] ]
\Tree
[.\node{IPb}{IP}
  \qroof{det}.PRONP 
  [.\node{Ibarb}{I'} [.Vfin \node{åpnet}{åpnet} ]
       \qroof{seg}.S ] ]

{\makedash{4pt}
\nodecurve[r]{Ibark}[t]{Ibarb}{38pt} 
}
\barnodeconnect[-10pt]{gaiGo}{åpnet} 
\nodecurve[t]{IPk}[t]{IPb}{15pt} 
 \caption{Umogleg samanstilling av funksjonsord mellom georgisk og norsk (bokmål)}
 \label{fig:ikkjesub}
\end{figure}

\subsection{Lik ordklasse?}
\label{sec-1.7}

Ulike språk leksikaliserer same konsept på ulike
måtar. \citet[s.~3]{cheung2002scg} skriv at det engelske ordet
\emph{fulfilment} meir naturleg blir omsett til eit verb på kinesisk. Det
same gjeld t.d. \emph{solitude} omsett til norsk. Eit georgisk
verbalsubstantiv (\emph{masdar}) kan bli omsett til eit verb i infinitiv på
norsk\footnote{Det georgiske verbalsubstantivet (\emph{masdar}) er i følgje
        \citet[kap.~2.5]{aronson1990grg} ein \emph{nominal} form, det kan i
        motsetning til norske verbalsubstantiv og engelske gerundium
        ikkje ta objekt, men kan ha modifiserande substantiv i
        genitiv. }. Slike skifte mellom ordklassar er svært vanlege i
omsetjing\footnote{\citet[Catford~(1965),~i][s.~61]{munday2001its} gir ein gjennomgang av
slike \emph{klasseskifte}, og andre typar omsetjingsskifte. }.

Me kan opne for ordklasseoverskridande lenkjer der det er samsvar
mellom visse \emph{trekk}, t.d. kan to predikerande ord lenkjast, eller to
«nominale» ord. Ein annan måte å gjere dette på er rett og slett å
krevje ein viss likskap i argumentstruktur. 


\subsection{Krav om lik argumentstruktur}
\label{sec-1.8}

\label{SEC:lik-argstr}

\citet{thunes2003eal} gir som nemnd eit krav om at \emph{predikat må ha tilsvarande semantiske argument} for å samanstillast.

Om det alltid er slik at to predikat har like mange argument, som kjem i
same rekkjefølgje i argumentstrukturen, vil det gjere den praktiske
oppgåva med å samanstille predikata, og argument med argument, mykje
enklare. Men kan me stille så sterke krav?

Sett at ein setning på språk 1 har ei \emph{at}-setning som adjunkt, medan
denne setninga på språk 2 er eit argument, og at desse setningane
ville vore samanstilte om dei opptrådde åleine. Om dei uttrykkjer same
proposisjon og \emph{speler same rolle i verbsituasjonen},
synest det naturleg å lenkje desse.  

Omsetjingsrelasjonar gir data for verbsituasjon, på eit meir generelt
grunnlag enn det me kan få frå einspråklege analysar åleine. Om me har
gode semantiske grunnar for å kalle ein deltakar i ein verbsituasjon
eit argument på eitt språk, vil dei same grunnane gjelde for
omsetjingsmessig korresponderande verb på andre språk. Ein kan då
nytte unionen over alle argument til korresponderande verb til å
karakterisere kva ein meiner med \emph{deltakarane i verbsituasjonen}. Syntaktiske forhold i språket kan sjølvsagt gi
grunnar til å \emph{ikkje} kalle dette eit argument (om det er mogleg å
finne akseptable syntaktiske grunnar for å kalle noko ein adjunkt
heller enn eit argument).
 
For å gjere dette konkret kan me sjå på setning 7 i MRS-suiten
\citep{mrs-suite}\footnote{Setningane i første og tredje linje i døma er direkte henta frå
MRS-suiten, med mindre anna er opplyst. }:

\exg.  abramsi brouns       daenajleva sigaretze, rom cvimda \\
      Abrams.NOM Brown.DAT vedde.3SG sigarett.om, at  regne.3SG.IMP \\
     `Abrams veddet en sigarett med Brown på at det regnet' 

I følgje LFG-parsen til desse setningane har hovudpredikata svært ulik
argumentstruktur\footnote{Analysane er henta 18. mai, 2009, frå
        \href{http://decentius.aksis.uib.no/logon/xle.xml}{http://decentius.aksis.uib.no/logon/xle.xml}, som implementerer
        LFG-grammatikkane frå ParGram-prosjektet \citep{butt2002pgp}. }. Det norske \emph{vedde} har \underline{fire} argument, medan
\emph{da-najleveba\} har \underline{to} (\emph{Abrams} og \emph{Browne} ), kor at-setninga på
norsk og \emph{rom cvimda} uttrykkjer same proposisjon og speler same rolle
i verbsituasjonen. Den engelske LFG-parsen av den tilsvarande setninga
(mine omsetjingar) gir \underline{tre} argument, \emph{with} blir her adjunkt, medan
den tyske grammatikken, som au har \underline{tre} argument, gjer \emph{at}-setninga
til adjunkt. I \Next nedanfor har eg representert dei omsetjingsmessig
korresponderande frasane i f-strukturane med dei norske omsetjingane
for å illustrere dette:

{\avmoptions{}
\ex. \label{vedde}
\a. Adams veddet en sigarett med Browne \hfill{} (norsk bokmål)\\ på at det regnet.\\
    $\\\begin{avm}\[pred & `{\bf{}vedde}<Abrams, sigarett, Browne, regne>' \\
                 adjunct & \{\}\]\end{avm}\\$
\b. abramsi brouns daenajleva sigaretze, rom cvimda. \hfill{} (georgisk)\\
    $\\\begin{avm}\[pred &  `{\bf{}da-najleveba}<Abrams, Browne, regne>'\\
    adjunct &  \{ \rm sigarett \}\]\end{avm}\\$ 
\c. Abrams hat mit Browne um eine Zigarette gewettet, \hfill{}(tysk)\\
    daß es regnet.\\
    $\\\begin{avm}\[pred & `{\bf{}wetten}<Abrams, sigarett>' \\
                  adjunct & \{ \rm Browne, sigarett \}\]\end{avm}\\$
\d. Abrams bet a cigarette with Brown that it was raining. \hfill{}(engelsk)\\
    $\\\begin{avm}\[pred & `{\bf{}bet}<Abrams, sigarett, regne>'\\
                  adjunct & \{ \rm Browne \}\]\end{avm}$

}

Om ein skal ha grammatikkane som datagrunnlag er det altså eit reellt
problem kva ein skal gjere med mangel på samsvar i
argumentstruktur. Om det alltid var fullstendig samsvar i
argumentstruktur, ville det vore trivielt å lenkje argument: viss to
korresponderande verb hadde tre argument, ville me lenkja det første
med det første, det andre med det andre og det tredje med det
tredje. Men om me har analysar som dei over, ser det ut til at me
treng bottom-up-informasjon om kva for adjunkt og argument som
samsvarer.

Det same gjeld forøvrig lenkjing av adjunkt til adjunkt. Adjunkt
plukker ut si eiga rolle der argument får rolla tildelt frå verbet, og
f-strukturane har ingen hierarkisk inndeling av desse slik me har for
verb og argument, dei er i staden representert som \emph{uordna mengder}.

\subsubsection{\textbf{TODO} Sitere eigen korpusundersøkjing av variasjon i arg-str?}
\label{sec-1.8.1}

Ei undersøkjing av den frasesamanstilte trebanken SMULTRON
\citep{samuelsson2006pap} mot LFG-grammatikkane for engelsk og tysk
fann at 2 av 15 korresponderande verbtoken\footnote{25 om ein inkluderer analysar kor minst eitt av argumenta
        ikkje hadde korrekt analyse (t.d. eit \textsc{PRO} der
        grammatikken burde funne eit substantiv). } for høgfrekvente
innhaldsverb fekk analysar kor argument korresponderte med adjunkt
\citep{unhammer2009aaa}.


\subsubsection{\textbf{TODO} Ulik følgje i argumentstruktur}
\label{sec-1.8.2}

I tillegg til at argument kan lenkjast til adjunkt, kan koreferente
argument ha ulik følgje i argumentstrukturen. Det er klart at me vil
lenkje objektet til \emph{gefallen} (eller bokmål: \emph{behage}) med subjektet
til \emph{like}, og omvendt.  Men rekkjefølgje i argumentstrukturane i
ParGram-prosjektet er ofte basert på syntaktisk funksjon heller enn
rolle, slik at eit verb som har opplevar som objekt og tema som
subjekt vil ha opplevar nedanfor tema i argumentstrukturen, medan ei
omsetjing av dette verbet kan ha tema nedanfor:

{\avmoptions{}
\ex. \a. sie$_j$ gefallen ihnen$_i$ \\
     $\begin{avm}\[pred & `{\bf{}gefallen}<de$_j$, de$_i$>' \]\end{avm}$
    $\\\\\leftrightarrow$\\
     \b. de$_i$ liker dem$_j$ \\
     $\begin{avm}\[pred & `{\bf{}like}<de$_i$, de$_j$>' \]\end{avm}$

}

Argumentstrukturane i \Last har omvendt intern følgje, og som vist ved
dette dømet er det heller ikkje noko f-strukturinformasjon me kunne
nytta til å sikre lenkjinga \emph{sie/dem} og \emph{ihnen/de}. Igjen ser det ut
til at bottom-up-informasjon trengst.


\paragraph{\textbf{TODO} Flytte til kapittel om metodar for å oppdage lenkjer?:}
\label{sec-1.8.2.1}

Kanskje me kan nytte data frå fleire førekomstar med andre subjekt
og objekt til å lære slike argumentstrukturalternasjonar?  Om me
observerer \emph{sie gefällt mir/jeg liker henne} vil me jo ha
f-strukturinformasjon som kan nyttast til å informere
argumentstrukturalternasjon (\emph{sie/henne} er hokjønn, etc.).

\subsection{\textbf{TODO} Konstruksjonar og komposisjonell inekvivalens}
\label{sec-1.9}

\xbar-teori føreset at det finst éi dotter i kvart ledd som kan
reknast som predikatet for dette leddet. Ei utfordring for
\xbar-baserte teoriar er då handsaming av \emph{komplekse predikat}. Desse
har fleire grammatiske element innanfor same ledd som alle bidrar med
«a non-trivial part of the information of the complex predicate»
\citep{alsina1997cp}. I LFG er det ein føresetnad at me berre har éin
\textsc{pred} ytterst i kvar f-struktur; ulike mekanismar har blitt
føreslått for å handsame dette fenomenet.

I omsette tekster kan me få eit analogt problem:

\ex. It can't be done \\
     Det lar seg ikke gjøre

Her vil ytre predikat i f-strukturen på norsk vere
`la<det$_1$,XCOMP>PRO', kor XCOMP[PRED `gjøre<NULL,det$_1$>NULL'].

På engelsk får me `can<XCOMP,it$_2$>', kor
XCOMP[PRED `do<NULL,it$_2$>']. 


Skal me lenkje orda \emph{can} og \emph{la}? På \emph{heile konstruksjonen} finn me
iallfall eit omsetjingsforhold:


\begin{center}
\begin{tabular}{lll}
 It can't be done                    &  Det lar seg ikke gjøre               &      \\
 can't be done                       &  lar seg ikke gjøre                   &      \\
 be done                             &  gjøre                                &  s?  \\
 \_{} can't be VPASS                 &  \_{} lar seg ikke VPASS              &  ??  \\
 \_$_{1}$ can \_$_{2}$ be VPASS$_3$  &  \_$_{1}$ lar seg \_$_{2}$ VPASS$_3$  &  ??  \\
\end{tabular}
\end{center}




(kan me få den siste generaliseringa frå trebanken?)


\subsection{evt. oppsummering}
\label{sec-1.10}




\bibliography{master}
\bibliographystyle{apacite}




















\end{document}