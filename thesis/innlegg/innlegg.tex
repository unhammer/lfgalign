% Created 2009-05-08 Fri 15:22
\documentclass[bigger]{beamer}
\usepackage[utf8]{inputenc}
\usepackage[T1]{fontenc}
\usepackage{hyperref}
\usepackage[english,nynorsk]{babel}
\usepackage{apacite} % after babel
\usepackage{natbib}
\usepackage{pslatex}

\usepackage{enumerate}
\usepackage{subfigure}
\usepackage{linguex}

\def\newblock{\hskip .11em plus .33em minus .07em} % for using bibtex with beamer 

\title{Syntaktisk fraselenking}
\author{Kevin Brubeck Unhammer\\ Universitetet i Bergen}
\date{\today}

\usetheme[orchid]{Singapore} 

\begin{document}
\maketitle

\begin{frame}\frametitle{Fraselenking}
    \begin{itemize}
    \item Finne korresponderande
      konstituentar / dependenseiningar / syntaktiske funksjonar / N-gram / ...
    \item Nyttig både for applikasjonar (maskinomsetjing) og
      korpuslingvistikk
    \item Data vanlegvis N-gramtabellar frå statistisk samanstilling,
      reint korpusbasert
    \item Formål vanlegvis maskinomsetjing
    \end{itemize}
\end{frame}

\begin{frame}\frametitle{Syntaktisk fraselenking, med \texttt{lfgalign}}
  \begin{itemize}
  \item Data: LFG-analysar, «kunnskapsbasert» lenking
  \item Formål: annotert trebank
  \end{itemize}
\end{frame}

\begin{frame}\frametitle{Vegkart}
  \begin{itemize}
  \item Krav til frasesamanstilling
  \item Implementasjon av \texttt{lfgalign}
  \item Evaluering
  \end{itemize}
\end{frame}

\begin{frame}\frametitle{Krav -- kva vil me lenkje?}
  \begin{itemize}
  \item \emph{Krav avheng av formål}
  \item Formål: trebankannotasjon
    \begin{itemize}
    \item presisjon viktigare enn dekning
    \item integrering med djupe analysar
    \end{itemize}
  \end{itemize}
\end{frame}

\begin{frame}\frametitle{Krav på ulike nivå}
  \begin{itemize}
  \item lenkjer mellom f-strukturar
  \item lenkjer mellom c-struktur
  \item lenkjer mellom ord % nedprioritert...
  \end{itemize}
\end{frame}

\begin{frame}\frametitle{Krav på ordnivå}
  Lenkja ord må ha LPT-korrespondanse:
  \begin{columns}
    \begin{column}{0.5\textwidth}
      \begin{itemize}
      \item LPT = Linguistically Predictable Translations
      \item bottom-up-informasjon (omsetjingsordbøker, giza++, ...)
      \item koreferente pronomen og substantiv/eigennamn er au LPT
      \end{itemize}
    \end{column}

    \begin{column}{0.5\textwidth}
      \begin{itemize}
      \item kaffi $=_{LPT}$ coffee
      \item kaffi $\neq_{LPT}$ tea
      \item han $=_{LPT}$ Joe
      \item kaffi $\neq_{LPT}$ Bob
      \item (kaffi $=_{LPT}$ Joe)
      \end{itemize}
    \end{column}
  \end{columns}
  % I \texttt{lfgalign} må lenkja ord ha lenkja f-strukturar
\end{frame}


\begin{frame}\frametitle{Krav på f-strukturnivå}
  Argumentstrukturar er ikkje alltid 1-1
  
  For å lenkje p og q:
  \begin{itemize}
  \item Alle argument av p skal finne LPT-korrespondanse i argument/adjunkt av q, og omvendt
  \item Adjunkt kan valfritt lenkjast til kvarandre
  \end{itemize}
\end{frame}

\begin{frame}\frametitle{Krav på c-strukturnivå}
  For å lenkje $n_s$ og $n_t$:
  \begin{itemize}
  \item $\phi$ av nodane må vere lenkja
  \item nodane må dominere same mengd med ordlenkjer
    \begin{itemize}
    \item Finn mengda av dominerte, f-strukturlenkja preterminale
      nodar, sjekk om $n_s$ og $n_t$ dominerer same mengd med slike
      lenkjer
    \end{itemize}
  \end{itemize}
\end{frame}


\begin{frame}\frametitle{Implementasjon}
  \begin{itemize}
  \item 1. Finn f-strukturlenkjer
  \item 2. Finn c-strukturlenkjer
  \end{itemize}
  (Men først må inndata vere LFG-parsa, setningslenkja og disambiguert)
\end{frame}

\begin{frame}\frametitle{Evaluering}
  \begin{itemize}
  \item 1. Manuell gjennomgang
  \item 2. Overlapp av lenkjer frå N-grambasert metode (\texttt{RIA})
  \end{itemize}
  % Uheldigvis ikkje så mykje materiale
\end{frame}

\begin{frame}\frametitle{Manuell evaluering}
  \begin{itemize}
  \item 
  \end{itemize}
\end{frame}



\begin{frame}
  \begin{center}
    {\huge Takk for merksemda!}
  \end{center}
\end{frame}

\begin{frame}\frametitle{Litteratur}
  \nocite{dyvik2009lmp}
  \bibliographystyle{apacite}
  \bibliography{master}
\end{frame}

\begin{frame}\frametitle{Lisensar}
  Denne presentasjonen kan distribuerast under lisensane
  GNU GPL, GNU FDL og CC-BY-SA.
  \begin{itemize}
  \item GNU GPL v. 3.0 \\
    \href{http://www.gnu.org/licenses/gpl.html}{http://www.gnu.org/licenses/gpl.html}
  \item GNU FDL v. 1.2 \\
    \href{http://www.gnu.org/licenses/gfdl.html}{http://www.gnu.org/licenses/gfdl.html}
  \item CC-BY-SA v. 3.0 \\
    \href{http://creativecommons.org/licenses/by-sa/3.0/}{http://creativecommons.org/licenses/by-sa/3.0/}
  \end{itemize}
\end{frame}

\end{document}